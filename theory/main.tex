\documentclass[12pt]{article}
\usepackage{amsmath}
\usepackage{physics}
\usepackage{bm}
\usepackage{amsfonts}

\title{Variational Principle Applied to the Quantum Harmonic Oscillator}
\author{}
\date{}

\begin{document}
\maketitle

\section{Introduction}

The variational principle is a powerful technique in quantum mechanics for approximating 
energy eigenvalues when the Schr\"odinger equation cannot be solved exactly.  
It guarantees that for any normalized trial wavefunction $\psi(a)$ depending on one or more 
parameters $a$, the expectation value of the Hamiltonian satisfies
\[
E(a) = \frac{\mel{\psi(a)}{\hat{H}}{\psi(a)}}{\braket{\psi(a)}{\psi(a)}} \ge E_0 ,
\]
where $E_0$ is the true ground state energy.  
By minimizing $E(a)$ with respect to the parameter(s), one obtains the closest possible 
approximation to the ground state energy within the chosen trial space.

\section{Quantum Harmonic Oscillator}

The Hamiltonian of the one-dimensional harmonic oscillator is
\[
\hat{H} = -\frac{\hbar^2}{2m}\frac{d^2}{dx^2} + \frac12 m\omega^2 x^2 .
\]
The exact ground state wavefunction is known to be Gaussian:
\[
\psi_0(x) = A e^{-\alpha x^2}, \qquad 
\alpha = \frac{m\omega}{2\hbar},
\]
with ground state energy
\[
E_0 = \frac{1}{2}\hbar\omega.
\]
In the variational method, if one chooses a Gaussian trial function, one is guaranteed to 
obtain the exact answer, because the true ground state is itself Gaussian.

\section{Trial Wavefunction and Expectation Value}

We choose a normalized trial wavefunction of the form
\[
\psi(x;a) = e^{-a x^2},
\]
where $a$ is a variational parameter.  
The energy expectation value is
\[
E(a) = 
\frac{
\displaystyle \int_{-\infty}^{\infty} 
\psi(x;a)\,\hat{H}\,\psi(x;a)\, dx
}{
\displaystyle \int_{-\infty}^{\infty} 
\left|\psi(x;a)\right|^2 dx
}.
\]

The kinetic energy contribution is
\[
T(a) =
-\frac{\hbar^2}{2m} \int_{-\infty}^{\infty}
\psi(x;a)\,\frac{d^2\psi(x;a)}{dx^2}\, dx ,
\]
and the potential energy term is
\[
V(a) =
\frac12 m\omega^2 \int_{-\infty}^{\infty} x^2 
\left|\psi(x;a)\right|^2 dx .
\]
The variational energy is then
\[
E(a) = T(a) + V(a).
\]
Minimizing $E(a)$ with respect to $a$ yields the best approximation.  
For the harmonic oscillator, the minimum occurs exactly at
\[
a = \frac{m\omega}{2\hbar},
\]
and the energy equals the analytic result $E = \frac12 \hbar \omega$.

\section{Explanation of the Python Code}

The provided Python/SymPy code implements the steps described above:

\begin{itemize}
    \item \textbf{Trial wavefunction:}  
    \[
    f(x) = e^{-a x^2}
    \]
    expressed in SymPy as \verb|f = exp(-a*x**2)|.
    
    \item \textbf{Kinetic energy integral:}  
    The second derivative is computed symbolically using \verb|diff| and inserted into
    \[
    T = \int f(x)\left(-\frac{\hbar^2}{2m}\frac{d^2 f}{dx^2}\right)\, dx .
    \]

    \item \textbf{Potential energy integral:}  
    The potential function
    \[
    V(x) = \frac12 m\omega^2 x^2
    \]
    is multiplied by $f(x)^2$ and integrated symbolically.

    \item \textbf{Total energy expectation value:}  
    The code computes
    \[
    E(a) = \frac{T + V}{\int f(x)^2 dx}
    \]
    and simplifies it.

    \item \textbf{Numerical evaluation:}  
    A range of values for the variational parameter $a$ is generated, 
    and the computed energy is plotted as a function of $a$.
    
    \item \textbf{Result:}  
    The graph shows a minimum at the analytic value $E = \frac12$ (in units $\hbar=m=\omega=1$), demonstrating the variational principle.
\end{itemize}

\section{Conclusion}

The variational method provides a systematic approximation to the ground state energy.  
By choosing a Gaussian trial wavefunction for the harmonic oscillator, the method 
reproduces the exact ground state energy because the Gaussian form is already exact.  
The Python implementation computes the expectation value symbolically and verifies 
the variational principle numerically by plotting $E(a)$ and showing its minimum.

\end{document}
